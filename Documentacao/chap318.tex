
\section{Backlog}


\subsection{Features}

\begin{minipage}{\linewidth}
\begin{itemize}
	\item O sistema será capaz de permitir aos utilizadores registar dados métricos, e aos que assim for designado consultá-los;
	\item O sistema será capaz de permitir aos utentes e aos acompanhantes responder a questionários e facultar as respostas aos médicos;
	\item O sistema será capaz de permitir que os utentes e acompanhantes tenham acesso à localização atual e esperada(futura) do utente. 
\end{itemize}
\end{minipage}


\subsection{Backlog items}
A imagem identificada neste documento como figura~\ref{fig:pb3180} representa a versão inicial do product backlog do projeto.

\begin{figure}[htb]
	\centering
	\includegraphics[width=0.9\linewidth]{img/Product\_backlog\_v3.png}  % largura percentual 
	\caption{Product Backlog}
	\label{fig:pb3180}
\end{figure}


\subsection{Tasks}

\begin{minipage}{\linewidth}
\begin{itemize}
	\item Tasks Sprint1 Milestone Análise de requisitos e modelação (figura~\ref{fig:task3181});
\end{itemize}
\end{minipage}

\begin{figure}[htb]
	\centering
	\includegraphics[width=0.9\linewidth]{img/milestone\_analise\_sprint1\_v2.png}  % largura percentual 
	\caption{Tasks Sprint1 Milestone Análise de requisitos e modelação}
	\label{fig:task3181}
\end{figure}

