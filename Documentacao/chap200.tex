
\chapter{Identificação do problema}

A gestão de um serviço de urgência é uma área das instituições de prestação de cuidados de saúde primários. \\ 
A sobrecarga deste serviço provoca um aumento dos tempos de espera porque o corpo clínico é constituído por recursos humanos finitos. Este problema pode afetar negativamente a instituição. \\ 
Uma das principais causas identificadas é a presença de um elevado número de pessoas com uma condição clínica de baixa gravidade, i.e. com uma classificação de acordo com a determinação do risco clínico conforme o protocolo de Protocolo Triagem Manchester \citep{Triagem2022}. 



\section{Proposta}
O Projeto OSLER propõe-se a fornecer um sistema complementar de apoio ao diagnóstico envolvendo o utente no processo de recolha de informações adicionais que possam estar relacionadas com o episódio de urgência.  



\subsection{Infraestrutura...}
\begin{itemize}
	\item A instituição fornece as informações à plataforma via RESTful HTTP Services. Inicia e consulta o processo com todas as informações atualizadas pelo utente e pelos serviços até o episódio de urgência ficar concluído;
	\item O ID do episódio de urgência é fornecido à plataforma juntamente com dados adicionais do utente (Nome, cor da triagem, data/hora de entrada, número de utente SNS);
	\item Todos os utilizadores do sistema podem aceder à plataforma num dispositivo com browser e com acesso Wi-Fi à rede da instituição.
\end{itemize}



\subsection{Desenvolvimento...}
\begin{itemize}
	\item A linguagem de programação \textbf{C\#} foi a escolhida para o desenvolvimento do back-end;
	\item Serão utilizados e aplicados os princípios \acrshort{softsolid} \citep{Naidu2021}.
\end{itemize}



\subsection{O sistema...}
\begin{itemize}
	\item Deve controlar o acesso inicial ao processo de cada utente;
	\item Deve considerar diferentes níveis de acesso para os diferentes tipos de utilizadores;
	\item Deve registar o ID do utilizador, data/hora em cada operação registada;
	\item Deve reter ocultando todas as informações apagadas e introduzidas pelos utilizadores na base de dados, só no final do processo é que se poderá fazer a limpeza desses dados. Os utilizadores com nível de acesso superior podem consultar e recuperar esses dados caso sejam necessários;
	\item Deve considerar os seguintes tipos de utilizadores (utente, acompanhante, triagem, enfermeiro, médico, sysadmin);
	\item Deve considerar um interface para diferentes idiomas para os utilizadores;
	\item Deve suportar a criação de diferentes questionários;
	\item Deve permitir atualização do local onde o utente se encontra durante todo o processo;
	\item No caso de o utente ter algum dispositivo de monitorização, o sistema deverá permitir a introdução de diversas leituras dos valores no intervalo de tempo indicado ou programado pelo técnico de saúde;
	\item Deve permitir que o acompanhante do utente preencha e/ou atualize as informações dos questionários;
	\item Deve permitir associar mais do que um episódio a mais do que um técnico de saúde;
	\item Deve permitir que qualquer um dos técnicos de saúde (com acesso,) possam atualizar e/ou acrescentar valores no processo do utente;
	\item Deve permitir que um médico possa consultar episódios anteriores do utente, fazendo a pesquisa pelo número de SNS;
	\item Deve manter o episódio com estado ativo até ser dada alta ao utente, neste caso o estado será "fechado";
	\item Deve registar todas as deslocações e locais de espera, utilizando uma ordem sequencial onde deve ser incluída data/hora;
	\item Deve registar data/hora para cada resposta do questionário;
\end{itemize}



\subsection{Controlo de acesso dos utilizadores}
\begin{itemize}
	\item Os utilizadores dos diversos serviços são adicionados pelo sysadmin;
	\item Qualquer um dos utilizadores autorizados de cada serviço pode consultar um episódio com o estado "aberto";
	\item A autenticação do utente e do acompanhante é feita utilizando o nº do episódio e um PIN para o acesso inicial, depois o controlo da sessão é feito internamente utilizando um token;
	\item Internamente o utente/acompanhante é identificado no sistema pelo ID do episódio de urgência. O ID de utilizador nos registos será calculado automaticamente utilizando um dígito "0"(zero) para o utente e "1"(um) para o acompanhante respetivamente;
\end{itemize}



