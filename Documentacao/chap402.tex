
\subsection{Estratégias seguidas}

O objetivo principal deste projeto na \acrshort{uctxt} de \acrlong{ucpds} é a aprendizagem e implementação do processo completo de desenvolvimento de uma aplicação de software em equipa utilizando metodologia ágil.

O foco principal foi e é o processo, mas também decidimos que a aplicação deveria ficar a funcionar dentro dos requisitos pré estabelecidos com todas as alterações e decisões tomadas durante as diversas reuniões da equipa.

As orientações iniciais foram feitas pelo \textit{Product Owner} e durante o desenvolvimento do projeto fizeram-se algumas modificações tendo sempre em consideração a parte funcional do projeto utilizando os diversos contributos dos membros da equipa. 

Para simplificar e otimizar o desenvolvimento foram criadas duas soluções, uma para a biblioteca e outra para o API,  que apesar de serem separadas a solução para o API é complementada pela solução da biblioteca através da camada \textbf{CTRL}.
A separação deveu-se ao facto de o ambiente de desenvolvimento assumir sempre que todos os objetos criados deveriam estar no API e isso é contra as regras de desenvolvimento de camadas que desenvolvemos. 

Por principio o API não deve expor os objetos internos da biblioteca. Seguimos a recomendação do Professor Nuno Rodrigues relativamente à forma de acesso e o controle dos dados, portanto a recomendação foi evitar a utilização do \textit{EntityFramework}, esta funcionalidade e controlo foi implementado na camada \textbf{.DAL} da biblioteca. 

A solução "\textbf{WebAppOSLERLib}" partilha o mesmo \textbf{\textit{namespace}} na biblioteca e foi construida com os seguintes projetos:
\begin{itemize}
	\item \textbf{.BO} - camada \textit{Business Objects} onde foram definidos todos os objetos do projeto;
	\item \textbf{.DAL} - camada \textit{Data Access Layer} onde foi implementado o sistema que permite que todos os objetos sejam persistentes, i.e. sejam gravados na base de dados;
	\item \textbf{.DB} - camada responsável pela comunicação com a base de dados, é utilizada exclusivamente pela camada \textbf{.DAL};
	\item \textbf{.Tools} - camada onde foram criadas as classes necessárias para exceções personalizadas para melhor controlo de erros, uma classe responsável por produzir ID genérico para todos os objetos e finalmente uma classe responsável pela leitura do ficheiro de configuração no formato XML com a informação necessária para ligar a biblioteca à base de dados e a chave de encriptação do \textit{token} no API;
	\item \textbf{.Consts} - camada onde foram criadas todas as constantes e estruturas necessárias para fornecer dados ao API e ao cliente, esta camada é utilizada tanto pela biblioteca como pelo API;
	\item \textbf{.CTRL} - camada com a classe que agrupa todas as classes da camada \textbf{.DAL}, fornece todos os comportamentos necessários para cada \textbf{DAO} (\textit{Data Access Object}). É utilizada pelo API e é criada uma instância desta classe a cada execução de cada \textit{endpoint} do API;
	\item \textbf{.PlayLib} - projeto responsável pela inicialização de alguns objetos principais da biblioteca na base de dados, também produz todos os dados adicionais para poderem ser utilizados nos diversos testes e experiências para o funcionamento do API durante todo o processo de desenvolvimento;
	\item \textbf{.UnitTest} - projeto onde são criados todos os testes unitários da biblioteca.
\end{itemize}

A solução "\textbf{WebAppOSLER}" implementa o API e comunica com a biblioteca através de instâncias da classe \textbf{AppCtrl} da camada "\textbf{.CTRL}", todo o acesso aos dados é \textbf{Thread-Safe} permitindo múltiplas ligações concorrentes à biblioteca e base de dados.




\subsection{Alteração do diagrama de classes}\label{novoClasseDiagram}

A implementação e desenvolvimento do projeto nesta fase criou a necessidade de otimizar e adaptar o diagrama de classes (figura~\ref{fig:cd3130}) à nossa estrutura de camadas utilizada para o desenvolvimento do software, conforme mencionado na \textbf{Milestone Análise de requisitos e modelação} (na página ~\pageref{oldClasseDiagram} na secção \ref{oldClasseDiagram}). Apresentamos nesta parte da documentação a versão completa do diagrama para refletir todas as camadas e as respetivas dependências. A figura~\ref{fig:cdNovoAll0} apresenta uma visão global das classes de todas as camadas com as suas dependências. As figuras ~\ref{fig:cdNovoAll1} e ~\ref{fig:cdNovoAll2} ilustram duas relações de dependência, a primeira entre as camadas \textbf{.DAL} e \textbf{.BO}, e a segunda entre as três camadas.


\begin{figure}[htb]
	\centering
	\includegraphics[width=0.9\linewidth]{img/Class\_Diagram\_v4\_v1\_camadas.png}  % largura percentual 
	\caption{Class Diagram (versão completa identificando a dependência entre camadas)}
	\label{fig:cdNovoAll0}
\end{figure}

\begin{figure}[htb]
	\centering
	\includegraphics[width=0.9\linewidth]{img/Class\_Diagram\_v4\_v1\_detalhe1.png}  % largura percentual 
	\caption{Class Diagram (detalhe dependência entre \textbf{.DAL} e \textbf{.BO})}
	\label{fig:cdNovoAll1}
\end{figure}

\begin{figure}[htb]
	\centering
	\includegraphics[width=0.9\linewidth]{img/Class\_Diagram\_v4\_v1\_detalhe2.png}  % largura percentual 
	\caption{Class Diagram (detalhe da dependência entre as 3 camadas)}
	\label{fig:cdNovoAll2}
\end{figure}


