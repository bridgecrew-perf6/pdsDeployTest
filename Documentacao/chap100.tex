
\chapter{Introdução}
De uma forma geral, o serviço de urgência é o que regista mais visitas do que os restantes serviços de uma instituição de prestação de cuidados de saúde primários. Normalmente a sobrecarga deste serviço provoca um aumento dos tempos de espera porque o corpo clínico é constituído por recursos humanos finitos. Este problema pode afetar negativamente a instituição. Uma das principais causas identificadas é a presença de um elevado numero de pessoas com uma condição clínica de baixa gravidade, i.e. com uma classificação de acordo com a determinação do risco clínico conforme o protocolo de \gls{triagemmanchester}. \\


\section{Contexto}
Este projeto é desenvolvido no âmbito da \acrshort{uctxt} de \acrshort{ucpds}. Pretende-se materializar todos os conhecimentos obtidos em diversas \acrshort{ucstxt} (\acrshort{ucaad}, \acrshort{ucams} e \acrshort{ucpes}) do semestre passado. 




\section{Objetivos}
O objetivo deste trabalho é dedicar todo o tempo da aula na implementação de um sistema de software. O sistema terá que ser implementado numa arquitetura com um \textbf{Web front-end} e um \textbf{back-end}. \\

\begin{itemize}
	\item O front-end desenvolvido na \acrshort{uctxt} de \acrlong{ucpw};
	\item O back-end desenvolvido na \acrshort{uctxt} de \acrlong{ucpds};
	\item A integração entre o front-end e o back-end será realizada através da uma API (\textit{\textbf{RESTful HTTP Services}});
	\item A metodologia \textbf{Scrum} será utilizada para o planeamento do projeto;
	\item O software \acrshort{softalm} escolhido para apoio à gestão deste projeto foi a plataforma Azure DevOps;
	\item Além deste documento, toda a informação será atualizada em documento excel na plataforma de E-learning da \acrshort{uctxt} de \acrlong{ucpds};
	\item Utilização de sistema de controlo de versões. Foi escolhido o Git que está incluído no espaço do projeto na plataforma Azure DevOps;
	\item A planificação do projeto será feita em 4 milestones para apresentar em aula (Especificação, vAlfa, vBeta e vRTW - Ready to Web).
\end{itemize}



\section{Estrutura do documento}
Este documento agrupa toda a documentação produzida pelo grupo de trabalho em todas as fases do projeto. \\

\begin{enumerate}
	\item Identificação do problema;
	\item Milestone Análise de requisitos e modelação. Especificação dos requisitos funcionais e não funcionais do sistema, \textit{mockups}, backlog completo e planeamento inicial de \textit{sprints};
	\item Milestone vAlfa. Requisitos funcionais implementados. Testar automaticamente todas as funcionalidades. Documentação de especificação deve incluir todas as alterações;
	\item Milestone vBeta. Sistema completamente implementado e funcional. Integração de todos os componentes. Identificar e apresentar aspetos funcionais e não funcionais a melhorar no sistema;
	\item Milestone vRTW. Versão final pronta a entrar em produção. Preparação de material promocional; 
	\item Conclusão deste documento;
\end{enumerate}

\subsection*{\textbf{NOTA}:}

Cada um dos capítulos da lista anterior inclui ainda dados referentes a todas as reuniões do grupo de trabalho. A parte alfanumérica que identifica cada reunião corresponde a cada \textit{milestone} do projeto:
\begin{itemize}
	\item \textbf{MM} = Milestone Análise de requisitos e modelação;
	\item \textbf{MA} = Milestone vAlfa;
	\item \textbf{MB} = Milestone vBeta;
	\item \textbf{MR} = Milestone vRTW. 
\end{itemize}


\section{Equipa}

A composição do grupo de trabalho, nome do projeto, nome da equipa e cargos foi finalizada na primeira reunião (identificada como reunião MM01 transcrita no capítulo \ref{reuniaoMM01} na página \pageref{reuniaoMM01}).\\[4mm]

\noindent \textbf{Nome do projeto}\\[1mm]
\noindent \rule{\linewidth}{0.4pt}
\noindent Projeto OSLER \\[4mm]

\noindent \textbf{Nome da equipa}\\[1mm]
\noindent \rule{\linewidth}{0.4pt}
\noindent Fuzzy Bit - Software Engineering \\[4mm]

\noindent \textbf{Membros e cargos}\\[1mm]
\noindent \rule{\linewidth}{0.4pt}
\noindent João Carlos Marques Pinto (20808) \textbf{Product Owner} + Development team member\\[1mm]
\noindent André Carvalho Mandim (21160) \textbf{Scrum Master} + Development team member\\[1mm]
\noindent António Augusto Fernandes Simões Pereira (21136) Development team member \\[1mm]
\noindent (*)Maria do Rosário Dias Figueiredo da Silva (21138) Development team member \\[1mm]
\noindent Rui Manuel da Silva Alves (15505) Development team member \\[2mm]

\noindent (*) decidiu sair do grupo a meio do primeiro milestone. \\[4mm]


