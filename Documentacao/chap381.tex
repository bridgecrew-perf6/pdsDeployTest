
\section{Reunião MM01}\label{reuniaoMM01}

\subsection*{Data:}
03-março-2022

\subsection*{Intervenientes:}
Grupo de trabalho

\subsection*{Âmbito:}
Constituir a equipa, identificar um nome para o grupo e para o projeto e iniciar o processo de análise. \\

\subsection*{Descrição:}
\textbf{*} Reunir os membros do grupo e atribuir os cargos. \\
João Carlos Marques Pinto (20808) Product Owner + Development team member\\
André Carvalho Mandim (21160) Scrum Master + Development team member\\
António Augusto Fernandes Simões Pereira (21136) Development team member \\
(*)Maria do Rosário Dias Figueiredo da Silva (21138) Development team member \\
Rui Manuel da Silva Alves (15505) Development team member \\
\noindent (*) decidiu sair do grupo a meio do primeiro milestone. \\

\textbf{*} Atribuir um nome ao grupo de trabalho. \\
O nome "Fuzzy Bit" foi o que reuniu consenso. \\

\textbf{*} Atribuir um nome ao projeto. \\
"Projeto OSLER" foi o que reuniu consenso. \\

\textbf{*} Inicio do processo de análise do projeto: \\
Depois de diversas ideias terem sido lançadas no debate inicial, chegamos a um consenso sobre a ideia de funcionamento da aplicação no processo inicial:

\begin{itemize}
	\item Receção faz entrada(identificação);
	\item Vai a triagem (avaliação, responde a questões -> atribui uma prioridade);
	\item Sistema recolhe mais dados (questionários, registo de valores, monitorização do doente pelo acompanhante);
	\item Consulta (médico além das perguntas normais, tem acesso aos dados que o doente introduziu).
\end{itemize}

Algumas \textit{features} identificadas:
\begin{itemize}
	\item O utente(acompanhante/enfermeiro) fornece informação útil para a consulta (entre a triagem e a consulta com o médico);
	\item O acompanhante tem acesso à localização (sala de espera ou serviço) do doente.
\end{itemize}

\noindent \rule{\linewidth}{0.4pt}
\newline
